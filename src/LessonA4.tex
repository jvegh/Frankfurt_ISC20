%%% This is the ISC-2020 Frankfurt 1st tutorial, A4

\MEchapter[Modern computing]{Modern computing}
\MESetListingFormat[basicstyle={\ttfamily\color{black}\normalsize}]{SystemC}

%\subsubsection{Modern computing I 45+5'}
%Some surprising parallels with
%studying the nature under extreme conditions and using
%computing under extreme conditions will be discovered and demonstrated
%that the common reason of the similarity of the surprising
%phenomena in those (apparently distant) field is
%the nonlinearity in extrapolating our experiences to 
%extreme parameter values. The goal for asking help from the science
%is that the consequent use of the ideas above leads to
%counter-intuitive conclusions and shocking results.
%The case is very similar to the revolution of physics
%more that hundred years ago: introducing the speed of light as 
%a speed limit or  that some measurement cannot be carried out
%\textit{at the same time} on a system.
%It will be shown that Amdahl's Law introduces a very similar
%performance limit for the parallelized sequentially working systems
%that cannot be exceeded and that the same processor cannot be
%equally good for single-thread performance and many-tasking performance. 


\MEsection[Modern computing]{Modern computing}
%%%%%%%%%%%%%%%%%%%%%%%%
\MEframe[shrink]{Modern computing}
{
\articleonly{} 

\articleonly{
}	
	The context
}% Lessons


\MEchapter[Science]{Modern physics and computing}



\MEsection{Role of science}
\MEframe{Why to apply science}
{% 
	The physicists were in regard to the apparent conflict between the classic and modern (relativistic and quantum) physics at the beginning of the $20^{th}$ century: \textit{under extreme conditions qualitatively different/strange behaviors may be encountered}, and for explaining them, the \textit{formerly unnoticed or neglected aspects are to be reconsidered}.
	Today computer scientists  experience \textbf{\textit{non-linearity of performance scaling}}  (among others).
	
	%\only<2->
	{The analogies want to call the attention to both that
		\textbf{\textit{under extreme conditions qualitatively different behavior may be encountered in both worlds}}, and
		that \textit{extending the theory with respect to formerly unnoticed or neglected aspects enable to explain
			the new phenomena}.}
	
	\only<2->{In contrast with the nature, \textbf{\textit{the technical world also enables making better computing through introducing enhanced computing paradigm and/or technology solutions}}.} 
}  

\MEframe{Why to apply science}
{% 
	The physical  world we live in is rather counter-intuitive to accept
	that  as we move towards unusual conditions, \textit{\textbf{adding of speeds behaves differently}}; 
	%when approaching the speed of light as well as that
	the energy becomes discontinuous; the momentum and the position of a particle cannot be measured accurately at the same time.
	
	%\only<2->
	{Similarly, in the world of computing, it is  counter-intuitive to accept that (i) -\textbf{\textit{in large parallelized sequential systems the payload performance deviates from the simple sum of the performance of the comprised single computers}}~\textbf{\cite{VeghPerformanceWall:2019}} (the phenomenon known as 'efficiency'~\cite{DifferentBenchmarks:2017})
		(ii) -the length of the clock period has noticeable effect on the performance~\textbf{\cite{VeghPerformanceWall:2019,VeghBrainAmdahl:2019}}
		(iii) -the 'built for speed' single processors with many registers, large cache, pipelining, accelerators, etc. considerably increase the latency time in many-many processor systems and  multi-tasking environment.}
}


\MEsection{The Light Speed}

\MEframe{Analogy with the special relativity}
{
	%\footnote{ %Mentor Graphics:
	%\url{http://blogs.scientificamerican.com/guest-blog/moore-s-law-and-the-future-of-solid-state-electronics/}}
	\maxsizebox{\textwidth}{!}
	{	
		\begin{tabular}{|p{190pt}|p{190pt}|}
			\hline
			\hline
			Physics & Computing\\
			\hline
			Adding of speeds &	Adding of performance\\
			\hline
			\textcolor{blue}{Classic} & \textcolor{blue}{Classic} \\
			{\large $ v(t) = \textcolor{blue}{t\cdot a}$}
			&
			{\large $ Perf_{total}(N) = \textcolor{blue}{N\cdot P_{single}}$}	\\
			\hline
			t = time &  N = number of cores \\
			\hline
			a = acceleration & $P_{single} = $ Single-performance\\
			\hline
			\only<2->{		n = optical density}
			&	\only<3->{\textcolor{red}{communication}}\\
			\hline
			\only<2->{c = Light Speed} &
			\only<3->{\textcolor{red}{$\alpha$ = parallelism}}\\
			\hline
			
			\only<2->{\textcolor{red}{ Modern (relativistic)}} &	\only<4->{{\textcolor{red}{Modern (Amdahl-aware)~\cite{VeghAlphaEff:2016}}}  }\\
			
			\only<2->				{\Large $ v(t) = \frac{t\cdot a}{\boxed{\textcolor{red}{\sqrt{1+\left(\frac{t\cdot a}{c/n}\right)^2}}}}$}
			&
			\only<4->	{\Large $ P(N) = \frac{N\cdot P_{single}}{\boxed{\textcolor{red}{N\cdot \left(1-\alpha\right)+\alpha}}}$}	
			\\
			\hline
			\hline
		\end{tabular}
	}
}

\MEframe{Analogy with the relativistic speed addition}
{%
	\maxsizebox{\textwidth}{!}
	{
		\begin{tabular}{cc}
			\only<1->
			{\maxsizebox{.5\textwidth}{!}
				{
					
					
					\def\LightSpeed{3.e8}	% m/s
					\def\Gravity{10.}		%m/s^2
					\def\Speed{x*\Gravity}
					\def\RelSpeedFactor{\Speed/(\LightSpeed/\Density)}
					\def\RelSpeed{\Speed/sqrt(1+\RelSpeedFactor*\RelSpeedFactor)}
					
					\def\RelSpeedFactorB{\Speed/(\LightSpeed/\Density/2)}
					\def\RelSpeedB{\Speed/sqrt(1+\RelSpeedFactorB*\RelSpeedFactorB)}
					
					\def\OneDay{86400}
					
					\begin{tikzpicture}%[scale=1.5]
					\begin{axis}[
					%  axis y line*=left,
					title={\huge Relativistic speed of body accelerated by 'g'},
					width=\textwidth,
					%	title style={at={(0.5,1.05)},anchor=north},
					%	title = {Relativistic speed of body accelerated by 'g'}, 
					xlabel=\huge $time(s)$,
					ylabel=\huge $speed (m/s)$,
					ymin=1e6, ymax=5e8,
					xmin=\OneDay, xmax=5e8,
					xmode=log,
					log basis x=10,
					ymode=log,
					log basis y=10,
					legend style={
						cells={anchor=west},
						legend pos={north west},
					},
					]
					\def\Density{1.}
					
					\addlegendentry{$v(t),~n=1$}
					\addplot[samples=501,domain=\OneDay:1e9,webgreen]
					{\RelSpeed} ;\label{plot_loo}\OneDay
					
					\def\Density{2.5}
					\addplot[samples=501,domain=\OneDay:1e9,webred]
					{\RelSpeed} ;\label{plot_loo}
					\addlegendentry{$v(t),~n=2.5$}
					
					\def\Density{5.}
					\addplot[samples=501,domain=\OneDay:1e9,webred]
					{\RelSpeed} ;\label{plot_loo}
					\addlegendentry{$v(t),~n=5$}
					
					\end{axis}
					
					
					\end{tikzpicture}
				}
				
			}&
			\only<2->
			{
				\maxsizebox{.5\textwidth}{!}
				{
					
					\def\alpha{(1-\beta)}
					
					\def\N{(x/0.0000001)}
					%	\def\PayloadPerformance{x/(\N*(1.-\alpha)+ \alpha)}%(\N*(1-\alpha)+\alpha)}
					\def\PayloadPerformance{x/(\alpha+100000000*x*\beta)}
					\def\betaN{1e-10}
					\def\betaA{1e-8}
					\def\betaB{1e-7}
					\def\betaC{1e-6}
					\def\betaD{1e-5}
					
					\begin{tikzpicture}%[scale=1.5]
					\begin{axis}[
					%  axis y line*=left,
					title={\huge Payload performances of N cores @100GFlops},
					width=\textwidth,
					%	title style={at={(0.5,1.05)},anchor=north},
					%	title = {Relativistic speed of body accelerated by 'g'}, 
					xlabel=\huge Nominal performance (EFlops),
					ylabel=\huge Payload performance (EFlops),
					ymin=1e-4, ymax=2,
					xmin=1e-4, xmax=2,
					xmode=log,
					log basis x=10,
					ymode=log,
					log basis y=10,
					legend style={
						cells={anchor=west},
						legend pos={north west},
					},
					]
					\def\beta{\betaN}
					
					\addlegendentry{1-$alpha=\betaN$}
					\addplot[samples=501,domain=1e-4:2,webred]
					{\PayloadPerformance} ;%\label{plot_0}
					\def\beta{\betaA}
					
					\addlegendentry{1-$alpha=\betaA$}
					\addplot[samples=501,domain=1e-4:2,webgreen]
					{\PayloadPerformance} ;%\label{plot_0}
					
					\def\beta{\betaB}
					
					\addlegendentry{1-$alpha=\betaB$}
					\addplot[samples=501,domain=1e-4:2,webgreen]
					{\PayloadPerformance} ;%\label{plot_0}
					
					\def\beta{\betaC}
					
					\addlegendentry{1-$alpha=\betaC$}
					\addplot[samples=501,domain=1e-4:2,webgreen]
					{\PayloadPerformance} ;%\label{plot_0}
					
					\def\beta{\betaD}
					
					\addlegendentry{1-$alpha=\betaD$}
					\addplot[samples=501,domain=1e-4:2,webblue]
					{\PayloadPerformance} ;%\label{plot_0}
					\end{axis}
					
					
					\end{tikzpicture}
					
			}}\\
		\end{tabular}
	}
	
	\only<1>{	A body accelerated by a constant gravitational force cannot exceed the speed of light.}
	
	\only<2>{	The performance of a parallelized sequential computing system 
		cannot exceed its specific 'speed of light'.
		
		The performance is sensitive to the amount of computation (including data length) and communication.
		
		Science is \textbf{not} limiting below EPlops performance.
	}
	\maxsizebox{\textwidth}{!}
	{
		\centering{
			\only<3>{\Large $ P(N) = \frac{N\cdot P_{single}}{\boxed{\textcolor{red}{N\cdot \left(1-\alpha\right)+\alpha}}}$
			}
			
			\only<4>{\Large $ P(N) = \frac{N\cdot P_{single}}{\boxed{\textcolor{red}{N}\cdot \left(1-\alpha_{Science}  
						\textcolor{red}{-\alpha_{Net} -\alpha_{Compute}} -\alpha_{Others}    \right)+\approx 1}}$
			}
		}	
	}
	
}
