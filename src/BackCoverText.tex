The stalling of the single-processor performance about two decades ago,
accompanied with the explosion-like growing demand on computing, put under
pressure all fields where some kind of performance increase can be expected.
Those fields include researching different materials, developing more or less different computing principles, developing interconnections with higher speed, developing and combining 
different kinds of accelerators, introducing reconfiguration both in computing and
in the interconnection, etc. 
During this, achieving higher numbers describing the new developments is a natural goal,
both from engineering and marketing point of view. However, as all engineering solutions
at least approached their limitations, enabled by laws of nature, the segregated
optimizations started to block each other. 

The present booklet attempts to put the base terms in order. It starts from the very beginning,
and brushes up principles known already decades ago. Some issues have been pointed out decades ago
and could be validly neglected up to now, but the respectable technical development 
caused the old issues to return in a technically different form. The booklet

\begin{center}
\huge\bfseries\sffamily\color{lime}`\LectureTitle'
\end{center}

wants to provide a systematic review of the basic ideas, merits, procedures, conventions, etc.
It provides a good starting point for understanding the performance limitations,
the understanding some mystic phenomena. Through providing a solid understanding
of the principles of the parallelized sequential processing, the careful reader can attain
a solid background for understanding the operation and performance limitations
of the infinite variety of such systems. The principles are carefully underpinned by
publicly available data.
