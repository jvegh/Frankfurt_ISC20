%%% This is the ISC-2020 Frankfurt 1st tutorial, A2

\MEchapter[Single-processor performance]{Limitations of parallelized sequential processing}
\MESetListingFormat[basicstyle={\ttfamily\color{black}\normalsize}]{SystemC}

%\subsubsection{Limitations of parallelized sequential processing 45+5'}
%The lesson interprets Amdahl's Law in its original 
%interpretation, and sets up a non-technical, general-purpose model.
%It visualizes the meaning of Amdahl's Law, and points out the existence of an inherent performance limit of the sequential operation.
%To describe the perfectness of the implementation of the implemented
%parallelization, introduces the parameter "effective parallelization".
%It will be shown that that the efficacy decrease with the 
%growing number of processors is caused by the principle of
%parallelization, rather than by some engineering imperfectness.
%The lesson introduces the "dark performance",
%i.e. the non-payload performance of parallelized sequential systems,
%and interprets the experience of "computing efficiency". Some surprising utilizations of Amdhal's Law as well as its abusing are also mentioned.


\MEsection[SPA]{Limitations of parallelized sequential processing}
%%%%%%%%%%%%%%%%%%%%%%%%

\MEsection[The math]{Amdahl's Law}

\MEframe{The mathematics of Amdahl's Law}
{
Amdahl's original intention, as it was expressed also in the title 
of his famous paper~\cite{AmdahlSingleProcessor67}, was to draw the attention to that parallizing single processors
implies serious performance limitations.
Amdahl only wanted to draw the attention to that when putting together
several single processors, and using \gls{SPA}, the available speed gain due to using large-scale computing capabilities \textit{has} a theoretical upper bound.
He also mentioned that data housekeeping (non-payload calculations)
causes some overhead, 
and  that \emph{the nature of that overhead appears to be sequential, independently of its origin}.
\index{Amdahl's law} \index{parallelization speedup}
}

%\MEframe{The mathematics of Amdahl's Law}
%{
%	\only<1>
%	{
%		\MEquote{Everyone knows Amdahl's Law, but quickly forgets it.}{Thomas Puzak, IBM, 2007}
%	}
%	
%	\only<2>
%	{
%		Usually,  Amdahl's law is expressed as 
%		\vspace{-.3\baselineskip}	
%		\begin{equation}
%		S^{-1}=(1-\alpha) +\alpha/N \label{eq:AmdahlBase}
%		\end{equation}
%		
%		\noindent where $N$ is the number of parallelized code fragments, 
%		$\alpha$ is the ratio of the parallelizable fraction to the total,
%		$S$ is the measurable speedup. 
%		
%		\vspace{-.3\baselineskip}	
%		\begin{equation}
%		\alpha = \frac{N}{N-1}\frac{S-1}{S} \label{equ:alphaeff}
%		\end{equation}
%		
%		When calculating speedup, one actually calculates
%		\vspace{-.3\baselineskip}	
%		\begin{equation}
%		S=\frac{(1-\alpha)+\alpha}{(1-\alpha)+\alpha/N} =\frac{N}{N(1-\alpha)+\alpha}
%		\end{equation}
%		hence  the \textit{efficiency}
%		\vspace{-.3\baselineskip}	
%		\begin{equation}
%		\boxed{E(\large N,\alpha)} = \frac{S}{N}=\boxed{\frac{1}{\textcolor{webred}{\Large N}(1-\alpha)+\alpha}}= \frac{R_{Max}}{R_{Peak}} \label{eq:soverk}
%		\end{equation}
%	}
%}
%
%\MEsection{The efficiency surface}
%\MEframe{Parallization efficacy as 2-parameter function}
%{
%	\MEfigure{fig/EffDependence2018LogA.pdf}
%	{	The efficiency of a distributed computing system, as defined by the first-order Amdahl's law. The surface is calculated according to Eq.~(\ref{eq:soverk}), the measured data are taken from the database \cite{TOP500:2017}.
%		"The efficency is a feature rather than a bug".
%	\pause
%	{”this decay in performance is not a fault of the
%		architecture, but is dictated by the limited parallelism”.\cite{ScalingParallel:1993}}
%}
%{fig:EfficiencySurface}{}
%{}
%}
%
%	\MEfigure{fig/ZettaflopsSandia}
%{What was expected in 2005}
%{Expectations2005}{}{.85}
%
%\MEsection[The model]{Amdahl's model}
%\MEframe{The model of parallel/sequential operation}
%{
%	\MEfigure{fig/AmdahlModelAnnotated.pdf}
%{Amdahl's model with annotations}
%{fig:AmdahlAnnotated}{}{.85}
%
%	\textbf{\textit{SW, \textcolor{red}{HW  and science} contribute to the sequential-only fraction.}}
%	
%	(The terms of the simple non-technical model need proper interpretation.)
%}

