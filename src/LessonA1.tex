%%% This is the ISC-2020 Frankfurt 1st tutorial, A

\MEchapter[Single-processor performance]{SPA: Around single-processor performance }
\MESetListingFormat[basicstyle={\ttfamily\color{black}\normalsize}]{SystemC}

\MEsection[SPA]{SPA: Around single-processor performance}
%%%%%%%%%%%%%%%%%%%%%%%%
\MEframe[shrink]{SPA: Around single-processor performance}
{
\articleonly{} 

\articleonly{
}	
	The context
}% Lessons


\MEframe[shrink]{SPA: Around single-processor performance}
{
\articleonly{It was early discovered that the age of conventional architectures was over ~\cite{AmdahlSingleProcessor67,GodfreyArchitecture1986};
the only question that remained open whether the “game is over”, too~\cite{ComputingPerformanceBook:2011}.} 
A slowdown or stalling of the growth of computing performance is expected according to the general experiences~\cite{ExponentialLawsComputing:2017}
and was really experienced for the performance of  both  single processors~\cite{LimitsOfLimits2014}
and large-scale parallelized sequential computing systems~\cite{ChinaExascale:2018,ToolingUpForExascale:2019}.
\articleonly{However, similarly to most of the basic limitations of computing,
even the limitations are limited~\cite{LimitsOfLimits2014}.
The limitations are topped by the issue of "dark performance"~\cite{Esmaeilzadeh:2015:AAP:2830689.2830693}: increasing simply the number of processors is not sufficient,
although the processor is a "free resource"~\cite{SpiNNaker:2013}.}
}


\MEframe[shrink]{SPA: Around single-processor performance}
{
\articleonly{When comes to parallel computing, the today’s technology is able to deliver not only many, but "too many" cores~\cite{TooManyCores2007}.}
Their computing performance, however, does not increase linearly with the number of cores~\cite{HillMulticoreAmdahl2008} 
("\textit{a trend that can't go on ad infinitum}"\cite{ExascaleGrandfatherHPC:2019}),
and only a few of them can be utilized simultaneously ~\cite{Computing_Dark_Silicon_2017}.  
\articleonly{Consequently, it is unreasonable to use high number of cores for such extreme system operation~\cite{Ungerer_MERASA:2010}.
Approaching the limits of the ”classic paradigm of computing” has rearranged the technology ranking~\cite{KeepingComputerIndustryInUS:2017}.}
"\textit{New ways of exploiting the silicon real estate need to be explored}"~\cite{ChandyParallelism:2009}.
}

\MEframe[shrink]{SPA: Around single-processor performance}
{
Not only the efficacy of computing is very low because of the different performance 
losses~\cite{InefficiencyHameed2010}, but more and more limitations come to the light~\cite{NeuromorphicComputing:2015}.
%--,ChandyParallelism:2009}. 
Not only that in extreme conditions computing quickly reaches  its performance limits~\cite{NeuromorphicComputing:2015,LimitsOfLimits2014},
but this performance degrades even more accelerated in systems comprising parallelized sequential processors~\cite{VeghPerformanceWall:2019}.
}

\MEframe[shrink]{SPA: Around single-processor performance}
{
The moon-shot of the limitless parallel processing performance is followed. Although even the exascale performance
is not yet achieved, already the $10^4$ times higher performance is planned\footnote{https://link.springer.com/journal/11714/19/10}.
\articleonly{Demonstrative failures (such as the supercomputers Gyoukou and Aurora, or the brain simulator SpiNNaker) come to the light, but the "gold rush" is going on. 
Even in the most prestigeous journals~\cite{StrechingSupercomputers:2017,Scienceexascale:2018}.}

It looks like that in the feasibility studies  an analysis
whether an inherent performance bound exists is done neither in USA~\cite{NSA_DOE_HPC_Report_2016} nor in  EU~\cite{EUActionPlan:2016}
nor in Japan~\cite{JapanExascale:2018} nor in China~\cite{ChinaExascale:2018}.
}