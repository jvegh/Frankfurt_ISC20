%%% This is the ISC-2020 Frankfurt 2nd tutorial, B2

\MEchapter[History of supercomputing]{History of performance of supercomputing}
\MESetListingFormat[basicstyle={\ttfamily\color{black}\normalsize}]{SystemC}


%\subsubsection{History of performance of supercomputing 45+5'}
%The rigorously verified and detailed database of TOP500 supercomputers
%provides an excellent starting point to draw conclusions.
%It will be demonstrated that the lesson learned: to increase
%the computing performance of the system the "effective parallelization" must also be enhanced, increasing the number of
%processors is not sufficient. It will be demonstrated that
%the "effective parallelization" is a proper merit and enables
%to compare the supercomputers from different ages, technology, manufacturers. The analysis makes clear that 
%the resulting performance stalled: the implementation technology 
%reached its limits.


\MEsection[History of supercomputing]{History of performance of supercomputing}
%%%%%%%%%%%%%%%%%%%%%%%%
\MEframe[shrink]{History of performance of supercomputing}
{
\articleonly{} 

\articleonly{
}	
	The context
}% Lessons
